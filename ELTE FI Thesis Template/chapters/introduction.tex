\chapter{Introduction}
\label{ch:introduction}

\section{History}

Recommendation systems are used to suggest top N recommendations to a given user. Different algorithms are used by different recommendation systems according to different data sources \cite{recommendation-engines}.

There are three main types of recommender systems:


\begin{enumerate}
\item Collaborative Filtering

The recommendations in collaborative filtering method are done by analyzing the information of user's behaviours \cite{recommendation-engines}. For example, if user 1 likes product A and B and user 2 likes product A and C, so there is a chance that user 1 will also like product C and user 2 will also like product B.


\item Content Based Filtering

The recommendations in content based filtering method are done based on the user's profile and description of the item. Items are recommended according to previous similar items liked or rated by the user \cite{recommendation-engines}. For example, if a user likes 'Titanic' then the movies starring 'Leonardo DiCaprio' or 'Romantic' movies can be recommended to this user.


\item Hybrid Recommendation Systems

This method is the combination of content based filtering and collaborative filtering methods. In some studies, it is claimed that hybrid approaches perform more accurately than actual approaches. The technique can be helpful to overcome the usual problems of recommender systems like sparsity problems or cold start. Netflix recommendation system is an example of a hybrid system \cite{recommendation-engines}.
\end{enumerate}

\section{Why do we need it?}

Recommendation systems are advantageous for both the users and the service providers. Many companies use a recommender system because they increase sales and improve customer experience. Recommendations reduce the time for users to find relevant content, and also give them new suggestions which they would not have searched for. The user might start liking and getting to know the service and spend more time on it, which sometimes leads the user to buy some additional items. Therefore, recommender systems help companies to get advantage over competitors and also decrease the chance of losing the user. Users might like a new product or movie that they were not aware of. Sometimes, the role of the recommender system is to show the users a whole new possibilities and products, which they would not search directly for themselves.

As Steve Jobs said: “A lot of times, people don’t know what they want until you show it to them” \cite{steve-jobs}.


\section{Motivation}

The massive increase of information and internet users have created a challenge of information overload. This increased the demand for recommendation systems higher than ever before. Current solutions use the method of prioritizing and personalizing the recommendation systems. As a result, it helps people to select items (e.g., movies, music, books) from a vast collection available on the internet. Many big companies use recommendation systems on their platform such as Netflix, Amazon, YouTube, Facebook, and many more. By using such systems people get to choose from a smaller collection of information according to the user's interest, preferences, or noticed behaviour about the item. Similarly a movie recommendation system helps users to watch movies with their recent interest. 

The main purpose of our movie recommendation system is to recommend similar movies to users according to the movie they searched. Our system will imply content based recommendation techniques to recommend movies. It will look for similar movies according to the movie user searches, it will first compare by cosine similarity the director, actors, and genres of the movies in the database with the searched movie. Then, it will compute the average between the user ratings and similarity scores of recommended movies, the final score determines the order. Finally, it will show the top 10 recommendations to users.

\section{Outline}


In Chapter \ref{ch:user}, we focus on how to use this program. A user will search a movie name to get recommendations based on that.

Chapter \ref{ch:developer} discusses the details of this program. The libraries and the model used in this project are explained. The datasets used in our model and analysis of the datasets are also discussed. All the definitions, functions, and components with code snippets, and descriptions are covered in this chapter. The example of comparison between with and without rating average is highlighted too.   

The final chapter (Chapter \ref{ch:conclusion}) is the thesis conclusion which will sum up all together and will also discuss improvements that could be done to give better recommendations.


